Italic capital letters are used for sets. Upper-case letters are used for elements. Quantities that are real-valued vector spaces are written in bold. Parameters are denoted by greek letters.

\begin{table}[]
    \begin{tabular}{ll}
    Symbol & Description \\
    $\mathcal{S}$ & The state space of the environment. \\
    $\mathcal{A}$ & The action space of the agent. \\
    $\mathbf{S}$ & A state. \\
    $\mathbf{S'}$ & A next state in a sequence. \\
    $\Lambda(\mathbf{S},\mathbf{A})$ & The transition function. Generates $\mathbf{S'}$. \\
    $\mathbf{A}$ & An action. \\
    $R$ & Reward value from the environment. \\
    $\hat R$ & Subjective reward value. \\
    $\mathcal{X}$ & The observation space. \\
    $\mathbf{X}$ & An observation. \\
    $T(\mathbf{S},\mathbf{A},\mathbf{S'},R,\mathbf{M})$ & The observation function. Generates $\mathbf{X}$. \\
    $\mathbf{M}$ & Memory (subjective). \\
    $f(\mathbf{\mathbf{X},M})$ & A learning function. Generates $\mathbf{M'}$. \\
    $\hat E$ & Subjective information value. \\
    $E(\mathbf{M},\mathbf{M'})$ & Information value function. Generates $\hat E$. \\
    $\pi$ & An action policy. Generates $\mathbf{A}$. \\
    $\pi_R$ & A policy to maximize reward ($R$, $\hat R$). \\
    $\pi_E$ & A policy to maximize information ($\hat E$). \\
    $\pi(\mathbf{S})$ & A deterministic action policy. \\
    $\pi(\mathbf{S}|\mathbf{A})$ & A stochastic action policy. \\
    $\Pi$ & A set of possible policies. A meta-policy. \\
    $J$ & A cost function. \\
    $J_E$ & A cost function for information value ($\hat E$). \\
    $J_R$ & A cost function for reward value ($R$, $\hat R$). \\
    $|.|$ & Denotes set size. \\
    $||.||$ & Denotes a vector norm. \\
    $\delta$ & Denotes a difference, between two scalars or vectors. \\
    $\nabla$ & Denotes a gradient. \\
    $\nabla^2$ & Denotes the Laplacian. \\
    $\alpha$ & Step-size (0,1]. \\
    $\beta$ & Softness parameter for softmax sampling (0, $\infty$). \\
    $\epsilon$ & Probability of random choice for e-greedy policy [0,1]. \\
    $\eta$ & Boredom bias [0,$\infty$). \\
    $\rho$ & Reward bias [0,$\infty$). \\
    &                          
    \end{tabular}
\end{table}
 